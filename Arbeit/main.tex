\documentclass[%
	BCOR12mm, % Bindekorrektur 12mm (geeignet für Lochung oder einfache Klebebindung)
	tablecaptionabove %
	]{scrbook}

\usepackage[ngerman]{babel} % für deutsche Texte
\usepackage[utf8]{inputenc} % UTF-8 Zeichen sind möglich
\usepackage{amssymb,amsmath} % Matheumgebung

\usepackage{natbib}

\usepackage{siunitx} % Temperatur \SI{5}{\degreeCelsius}
\usepackage{chemfig} % chemische Sturkturformeln darstellen http://www.suedraum.de/latex/stammtisch/strukturformeln.pdf

%--------------------------------------------------------------------------------
\begin{document}
%--------------------------------------------------------------------------------
%	DECKBLATT
%--------------------------------------------------------------------------------
\titlehead{\Large Karlsruher Institut für Technologie (KIT) \hfill Institut} % Universität, Institut, ...
\subject{Abschlussarbeit} % Art der Arbeit (Bachelor-, Master-, Diplomarbeit, Dissertation)
\title{Titel}
\subtitle{Untertitel}
\author{Autor} % mehrere Autoren mit \and trennen
%\date{} Datum wird automatisch auf \today gesetzt - für anderes Datum in die {} schreiben
\publishers{} % Aufgabensteller und Betreuer, getrennt mit \and
% \thanks{Dank an \dots} % Fußnote
\maketitle
%--------------------------------------------------------------------------------
%--------------------------------------------------------------------------------
%	ERKLÄRUNG - INHALTSVERZEICHNIS - SYMBOLVERZEICHNIS
%--------------------------------------------------------------------------------
\frontmatter
%--------------------------------------------------------------------------------
\chapter{Erklärung} % (fold)
\label{cha:erklaerung}

% chapter erklärung (end)
%--------------------------------------------------------------------------------
\tableofcontents
%--------------------------------------------------------------------------------
\chapter{Symbolverzeichnis} % (fold)
\label{cha:symbolverzeichnis}

% chapter symbolverzeichnis (end)

%--------------------------------------------------------------------------------
%	INHALT
%--------------------------------------------------------------------------------
\mainmatter
%--------------------------------------------------------------------------------
\chapter{Einleitung} % (fold)
\label{cha:einleitung}

% chapter einleitung (end)
%--------------------------------------------------------------------------------

\chapter{Theoretische Grundlagen} % (fold)
\label{cha:theoretische_grundlagen}
\cite{pradipasena2007temperature}
\citep{gaukel2004untersuchungen}
\citet{young1957d}
% chapter theoretische_grundlagen (end)
%--------------------------------------------------------------------------------

\chapter{Material und Methoden} % (fold)
\label{cha:material_und_methoden}

% chapter material_und_methoden (end)
%--------------------------------------------------------------------------------

\chapter{Ergebnisse und Diskussion} % (fold)
\label{cha:ergebnisse_und_diskussion}

% chapter ergebnisse_und_diskussion (end)
%--------------------------------------------------------------------------------

\chapter{Zusammenfassung und Ausblick} % (fold)
\label{cha:zusammenfassung_und_ausblick}

% chapter zusammen_und_ausblick (end)

%--------------------------------------------------------------------------------
%	LITERATURVERZEICHNIS - ANHANG
%--------------------------------------------------------------------------------
\bibliographystyle{alpha}
\bibliography{../literatur}
\appendix
\listoffigures
\listoftables
%--------------------------------------------------------------------------------
\end{document}