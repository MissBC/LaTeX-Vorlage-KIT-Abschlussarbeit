\documentclass[ %
	DIV=15,
	BCOR=14mm, % Bindekorrektur 12mm (geeignet für Lochung oder einfache Klebebindung)
	parskip=half, % Absätze mit Abstand von einer halben Zeilenhöhe
	headsepline, % Horizontale Linie zwischen Kopfzeile und Inhalt
	cleardoublepage=empty, % Vakatseiten ohne Kopf- und Fußzeile
	tablecaptionabove, % Tabellenüberschriften, aber Bildunterschriften
	toc=bib, % Literaturverzeichnis im Inhaltsverzeichnis aufführen
	toc=listofnumbered, % Abbildungs- und Tabellenverzeichnis im Inhaltverzeichnis aufführen
	listof=leveldown, % Abbildungs- und Tabellenverzeichnis Überschriften als Section statt Chapter
	numbers=noendperiod % Kein abschließender Punkt bei Kapitel-Zählung
	]{scrbook}

\usepackage[utf8]{inputenc} % UTF-8 Zeichen sind möglich
\usepackage[T1]{fontenc} % utf8-Kompatible Schrifttypen (Type One) erlauben

\usepackage[lf, minionint, loosequotes]{MinionPro}
\usepackage[lf]{MyriadPro}
\usepackage{courier}

\setkomafont{caption}{\small\itshape}
\setkomafont{captionlabel}{\normalfont\sffamily\bfseries}
\setkomafont{pageheadfoot}{\normalfont}

\usepackage{scrpage2}
\pagestyle{scrheadings}
\clearscrheadfoot
\ohead{\pagemark}
\ihead{\headmark}

\usepackage[ngerman]{babel} % für deutsche Texte
\usepackage[ngerman]{translator}

\usepackage{natbib}

\usepackage{pgfplots}
\usepackage{tikz}
\usetikzlibrary{backgrounds, calc}
\usepackage{color}

\usepackage{booktabs}
\usepackage[
	table-parse-only,
	detect-all,
	decimalsymbol = comma,
	list-units = single,
	range-units = single]{siunitx} % Temperatur \SI{5}{\degreeCelsius}
	\SendSettingsToPgf
\usepackage{mhchem}
\usepackage{chemfig} % chemische Sturkturformeln darstellen http://www.suedraum.de/latex/stammtisch/strukturformeln.pdf

\usepackage{listings}
\definecolor{codegreen}{rgb}{0,0.6,0}
\definecolor{codegray}{rgb}{0.5,0.5,0.5}
\definecolor{codepurple}{rgb}{0.58,0,0.82}
\definecolor{backcolour}{rgb}{0.97,0.97,0.97}
 
\lstdefinestyle{mystyle}{
    backgroundcolor=\color{backcolour},   
    commentstyle=\color{codegreen},
    keywordstyle=\color{magenta},
    numberstyle=\ttfamily\tiny\color{codegray},
    stringstyle=\color{codepurple},
    basicstyle=\footnotesize\ttfamily,
	xleftmargin=0.1\textwidth,
	xrightmargin=0.1\textwidth,
	columns=fullflexible,	
    breakatwhitespace=false,         
    breaklines=true,                 
    captionpos=b,                    
    keepspaces=true,                 
    numbers=left,                    
    numbersep=5pt,                  
    showspaces=false,                
    showstringspaces=false,
    showtabs=false,
}
 
\lstset{style=mystyle}

\usepackage{hyperref}

\definecolor{kit-green100}{cmyk}{1,0,.6,0}
\definecolor{kit-green70}{cmyk}{.7,0,.42,0}
\definecolor{kit-green50}{cmyk}{.5,0,.3,0}
\definecolor{kit-green30}{cmyk}{.3,0,.18,0}
\definecolor{kit-green15}{cmyk}{.15,0,.09,0}

\definecolor{kit-blue100}{cmyk}{1,.5,0,0}
\definecolor{kit-blue70}{cmyk}{.56,.53,0,0}
\definecolor{kit-blue50}{cmyk}{.4,.25,0,0}
\definecolor{kit-blue30}{cmyk}{.24,.15,0,0}
\definecolor{kit-blue15}{cmyk}{.12,.075,0,0}

\definecolor{kit-yellow100}{cmyk}{0,.05,1,0}
\definecolor{kit-yellow70}{cmyk}{0,.035,.7,0}
\definecolor{kit-yellow50}{cmyk}{0,.025,.5,0}
\definecolor{kit-yellow30}{cmyk}{0,.015,.3,0}
\definecolor{kit-yellow15}{cmyk}{0,.0075,.15,0}

\definecolor{kit-orange100}{cmyk}{0,.45,1,0}
\definecolor{kit-orange70}{cmyk}{0,.315,.7,0}
\definecolor{kit-orange50}{cmyk}{0,.225,.5,0}
\definecolor{kit-orange30}{cmyk}{0,.135,.3,0}
\definecolor{kit-orange15}{cmyk}{0,.0675,.15,0}

\definecolor{kit-red100}{cmyk}{.25,1,1,0}
\definecolor{kit-red70}{cmyk}{.175,.7,.7,0}
\definecolor{kit-red50}{cmyk}{.125,.5,.5,0}
\definecolor{kit-red30}{cmyk}{.075,.3,.3,0}
\definecolor{kit-red15}{cmyk}{.0375,.15,.15,0}

\newcommand{\schwamm}[3]{\ce{#1} \num{0,#2} \SI{#3}{}}

\newcommand{\figuremessurements}[3]{
\begin{figure}[htbp]
\begin{center}
	\graphicmessurements{#1}{#2}{#3}
	\caption[Feuchter Druckverlust des Schwamms \schwamm{#1}{#2}{#3} über der Gasgeschwindigkeit, doppellogarithmisch aufgetragen.]{Feuchter Druckverlust des Schwamms \schwamm{#1}{#2}{#3} über der Gasgeschwindigkeit, doppellogarithmisch aufgetragen. Stapelhöhe der Packung ist \SI{200}{\milli\m}. Verbindungslinien dienen ausschließlich der optischen Führung.}
	\label{fig:#1#2#3}
\end{center}
\end{figure}
}

\newcommand{\graphicmessurements}[3]{
	\begin{tikzpicture}
		\begin{loglogaxis}[
			log ticks with fixed point,
			grid=minor,
			xmin=0.1,ymin=10,
			xmax=1.3,ymax=7000,
		    xlabel=$u_g \text{ in } \si{\m\per\s}$,
		    ylabel=$\frac{\Delta p}{L} \text{ in } \si{\Pa\per\m}$,
			legend style={font=\small},
			legend pos=outer north east,
			scatter/classes={
				a={kit-blue100},
				b={kit-green100},
				c={kit-yellow100},
				d={kit-orange100},
				e={kit-red100}}]
			\addplot[
				scatter,
				scatter src=explicit symbolic,
				mark=*,
				smooth]
				table[header=false,x index=1,y index=0,meta index=3]{../data/#1_#2_#3.txt};
			\node[anchor=north west,draw=gray,fill=white] at (rel axis cs:0.05,0.95) {\schwamm{#1}{#2}{#3}};
			\legend{$u_l \approx \SI{5,7}{\milli\m\per\s}$,$u_l \approx \SI{4,2}{\milli\m\per\s}$,$u_l \approx \SI{2,8}{\milli\m\per\s}$,$u_l \approx \SI{1,4}{\milli\m\per\s}$,trocken}
		\end{loglogaxis}
	\end{tikzpicture}
}

%% ----------------------------------
%% |  Style of appendix numbering   |
%% ----------------------------------
\renewcommand\appendix{\par
	\addchap{\appendixname}
	\setcounter{section}{0}% 
	\setcounter{subsection}{0}% 
	\setcounter{figure}{0}%
	\renewcommand\thesection{\Alph{section}}% 
	\renewcommand\thefigure{\Alph{section}.\arabic{figure}} 
	\renewcommand\thetable{\Alph{section}.\arabic{table}}}
%% --- End of appenix numbering style ---

%--------------------------------------------------------------------------------
\begin{document}
%--------------------------------------------------------------------------------
%	DECKBLATT
%--------------------------------------------------------------------------------
%!TEX root = /Users/Cornelia/Documents/LaTex/Vorlagen/LaTeX-Vorlage-KIT-Abschlussarbeit/Arbeit/main.tex

%--------------------------------------------------------------------------------
%	DECKBLATT
%--------------------------------------------------------------------------------

\definecolor{grau}{rgb}{0.85,0.85,0.85} %RGB: 217R 217G 217B



\newcommand{\diameter}{5}
\newcommand{\xone}{-20} % -30 wenn keine Bindung/Lochung vorgesehen ist
\newcommand{\xtwo}{164}
\newcommand{\yone}{28}
\newcommand{\ytwo}{-251} % -259 wenn kein Text in die Fußzeile soll

\begin{tikzpicture}[ 
	show background rectangle, 
	background rectangle/.style={fill=grau},
	remember picture, 
	overlay]
	
\fill[color=white](\xone mm, \yone mm)  -- (\xtwo mm, \yone mm) arc (90:0:\diameter mm)   -- (\xtwo mm + \diameter mm , \ytwo mm) 	-- (\xone mm + \diameter mm , \ytwo mm) arc (270:180:\diameter mm)	-- (\xone mm, \yone mm);

\node[xshift=38.5mm,yshift=-17.5mm] at (current page.north west){
	\includegraphics[width=33mm, height=15mm]{../logos/KITLogo_RGB.pdf}
	};
\node[xshift=-26.5mm,yshift=-17.5mm] at (current page.north east){
	\includegraphics[height=15mm]{../logos/LVTLogo.pdf}
	};		 
\node [xshift=5mm,yshift=5mm] at (current page.center)[text width=\textwidth]
	{\titlehead{\Large \center Karlsruher Institut für Technologie (KIT)} % Universität, Institut, ...
	\subject{Abschlussarbeit} % Art der Arbeit (Bachelor-, Master-, Diplomarbeit, Dissertation)
	\title{Titel}
	\subtitle{Untertitel}
	\author{Autor}
	% \publishers{Aufgabensteller \\Betreuer}  % Aufgabensteller / Betreuer
	%\date{} %Datum wird automatisch auf \today gesetzt - für anderes Datum in die {} schreiben
	\maketitle };
	
\node [xshift=90mm,yshift=6.5mm] at (current page.south west)[text width=\textwidth]
	{Karlsruher Institut für Technologie (KIT)};
\node [xshift=42mm,yshift=6.5mm] at (current page.south east)[text width=\textwidth]
	{www.kit.edu};
\end{tikzpicture}
%--------------------------------------------------------------------------------
%--------------------------------------------------------------------------------
%	ERKLÄRUNG - INHALTSVERZEICHNIS - SYMBOLVERZEICHNIS
%--------------------------------------------------------------------------------
\frontmatter
%--------------------------------------------------------------------------------
\chapter*{Erklärung} % (fold)
\label{cha:erklaerung}

% \vspace*{4cm}
% {\Large\bfseries Erklärung}\par\medskip
%% ==================

Ich versichere hiermit wahrheitsgemäß, die Arbeit selbstständig verfasst und keine anderen als die angegebenen Quellen und Hilfsmittel benutzt, die wörtlich oder inhaltlich übernommenen Stellen als solche kenntlich gemacht und die Satzung des Karlsruher Instituts für Technologie (KIT) zur Sicherung guter wissenschaftlicher Praxis in der jeweils gültigen Fassung beachtet zu haben.

\bigskip
\bigskip
\hspace*{2cm} Karlsruhe, den \date{\today} \hspace*{0.5cm}\hrulefill \\
\hspace*{9.5cm} \author{Max Mustermann}\\

% chapter erklärung (end)
%--------------------------------------------------------------------------------
\tableofcontents
%--------------------------------------------------------------------------------
\chapter{Symbolverzeichnis} % (fold)
\label{cha:symbolverzeichnis}

% chapter symbolverzeichnis (end)

%--------------------------------------------------------------------------------
%	INHALT
%--------------------------------------------------------------------------------
\mainmatter
%--------------------------------------------------------------------------------
\chapter{Einleitung} % (fold)
\label{cha:einleitung}

Damit Ihr indess erkennt, woher dieser ganze Irrthum gekommen ist, und weshalb man die Lust anklagt und den Schmerz lobet, so will ich Euch Alles eröffnen und auseinander setzen, was jener Begründer der Wahrheit und gleichsam Baumeister des glücklichen Lebens selbst darüber gesagt hat. Niemand, sagt er, verschmähe, oder hasse, oder fliehe die Lust als solche, sondern weil grosse Schmerzen ihr folgen, wenn man nicht mit Vernunft ihr nachzugehen verstehe. Ebenso werde der Schmerz als solcher von Niemand geliebt, gesucht und verlangt, sondern weil mitunter solche Zeiten eintreten, dass man mittelst Arbeiten und Schmerzen eine grosse Lust sich zu verschaften suchen müsse. Um hier gleich bei dem Einfachsten stehen zu bleiben, so würde Niemand von uns anstrengende körperliche Uebungen vornehmen, wenn er nicht einen Vortheil davon erwartete. Wer dürfte aber wohl Den tadeln, der nach einer Lust verlangt, welcher keine Unannehmlichkeit folgt, oder der einem Schmerze ausweicht, aus dem keine Lust hervorgeht? \SIlist{1;2;3}{\metre} \\ \SIrange{1}{10}{\degreeCelsius}

\begin{table}[htbp]
	\caption{Eingesetzte Schwämme anhand ihres Materials, ihrer Porosität, Porenzahl, Höhe und spezifischen Oberfläche.}
	\label{tab:schwaemme}
		
	\begin{center}
	\begin{tabular}{ccSSSSS}
		\toprule
		\multicolumn{1}{c}{Hersteller} & \multicolumn{1}{c}{Material} & \multicolumn{1}{c}{$\psi_N$} & \multicolumn{1}{c}{Porenzahl} & \multicolumn{1}{c}{Höhe} & \multicolumn{1}{c}{$\psi^*$} & \multicolumn{1}{c}{${S_v}^*$}\\
		& & & \multicolumn{1}{c}{in} & \multicolumn{1}{c}{in \si{\milli\m}} & & \multicolumn{1}{c}{in \si{\square\m\per\cubic\m}}\\
		\midrule
		Vesuvius & \ce{Al2O3} & 0.75 & 10 & 50 & 0.688 & 640\\
		Vesuvius & \ce{Al2O3} & 0.85 & 10 & 50 & 0.812 & 630\\
		Vesuvius & \ce{Al2O3} & 0.85 & 20 & 50 & 0.813 & 970\\
		Vesuvius & \ce{Al2O3} & 0.85 & 30 & 50 & 0.793 & 1330\\
		Erbicol & \ce{SiSiC} & 0.88 & 10 & 25 & 0.865 & 480\\
		Letschert & \ce{Silikat} & 0.91 & 10 & 100 & 0.878 & 700\\
		\bottomrule
		\multicolumn{7}{l}{${}^*$ Werte aus \citet[][S. 155 und 157]{Grose2011Uber-keramische}}
	\end{tabular}
	\end{center}
\end{table}

Dagegen tadelt und hasst man mit Recht Den, welcher sich durch die Lockungen einer gegenwärtigen Lust erweichen und verführen lässt, ohne in seiner blinden Begierde zu sehen, welche Schmerzen und Unannehmlichkeiten seiner deshalb warten. Gleiche Schuld treffe Die, welche aus geistiger Schwäche, d.h. um der Arbeit und dem Schmerze zu entgehen, ihre Pflichten verabsäumen. Man kann hier leicht und schnell den richtigen Unterschied treffen; zu einer ruhigen Zeit, wo die Wahl der Entscheidung völlig frei ist und nichts hindert, das zu thun, was den Meisten gefällt, hat man jede Lust zu erfassen und jeden Schmerz abzuhalten; aber zu Zeiten trifft es sich in Folge von schuldigen Pflichten oder von sachlicher Noth, dass man die Lust zurückweisen und Beschwerden nicht von sich weisen darf. Deshalb trifft der Weise dann eine Auswahl, damit er durch Zurückweisung einer Lust dafür eine grössere erlange oder durch Uebernahme gewisser Schmerzen sich grössere erspare.

\begin{figure}[htbp]
\begin{center}
	\chemfig{*6((-HO)-=-(-(<[::60]OH)-[::-60]-[::-60,,,2] HN-[::+60]CH_3)=-(-HO)=)}
	\caption{Darstellung des Adrenalin-Moleküls mittels chemfig}
\end{center}
\end{figure}

% chapter einleitung (end)
%--------------------------------------------------------------------------------

\chapter{Theoretische Grundlagen} % (fold)
\label{cha:theoretische_grundlagen}

\section{Lorem ipsum dolor} % (fold)
\label{sec:lorem_ipsum_dolor}

Lorem ipsum dolor sit amet, consetetur sadipscing elitr, sed diam nonumy eirmod tempor invidunt ut labore et dolore magna aliquyam erat, sed diam voluptua. At vero eos et accusam et justo duo dolores et ea rebum. Stet clita kasd gubergren, no sea takimata sanctus est Lorem ipsum dolor sit amet. Lorem ipsum dolor sit amet, consetetur sadipscing elitr, sed diam nonumy eirmod tempor invidunt ut labore et dolore magna aliquyam erat, sed diam voluptua. At vero eos et accusam et justo duo dolores et ea rebum. Stet clita kasd gubergren, no sea takimata sanctus est Lorem ipsum dolor sit amet. Lorem ipsum dolor sit amet, consetetur sadipscing elitr, sed diam nonumy eirmod tempor invidunt ut labore et dolore magna aliquyam erat, sed diam voluptua. At vero eos et accusam et justo duo dolores et ea rebum. Stet clita kasd gubergren, no sea takimata sanctus est Lorem ipsum dolor sit amet.

\figuremessurements{Al2O3}{85}{10}

Duis autem vel eum iriure dolor in hendrerit in vulputate velit esse molestie consequat, vel illum dolore eu feugiat nulla facilisis at vero eros et accumsan et iusto odio dignissim qui blandit praesent luptatum zzril delenit augue duis dolore te feugait nulla facilisi. Lorem ipsum dolor sit amet, consectetuer adipiscing elit, sed diam nonummy nibh euismod tincidunt ut laoreet dolore magna aliquam erat volutpat.

Kurvenintegral (normale Anordnung der Grenzen):
  \[
    \int_C U = \int_a^b U \big(\vec{r}\,(t)\big)\,|\vec{r}\,'(t)|\,dt
  \]
  Arbeitsintegral (Verwendung von \verb|\limits|):
  \[
    \int\limits_C \vec{F} \cdot d\vec{r} 
      =\int\limits_a^b \vec{F} \big(\vec{r}\,(t)\big)
      \cdot \vec{r}\,'(t)\,dt
  \]
  Fluss durch eine Sphäre mit Radius $a$ (Mehrfachintegral jeweils mit Grenzen):
  \[
    \int_0^\pi \int_0^{2\pi} F_r a^2 \sin\vartheta \, d\varphi d\vartheta
  \]
  Satz von Green (Mehrfachintegrale über Bereiche mit \verb|\limits|):
  \[
    \iint\limits_S (U \operatorname{grad} W)\cdot d\vec{S} 
    =\iiint\limits_V (\operatorname{grad} U\cdot 
     \operatorname{grad} W +U\Delta W)\,dV
  \]

Ut wisi enim ad minim veniam, quis nostrud exerci tation ullamcorper suscipit lobortis nisl ut aliquip ex ea commodo consequat. Duis autem vel eum iriure dolor in hendrerit in vulputate velit esse molestie consequat, vel illum dolore eu feugiat nulla facilisis at vero eros et accumsan et iusto odio dignissim qui blandit praesent luptatum zzril delenit augue duis dolore te feugait nulla facilisi. 

Nam liber tempor cum soluta nobis eleifend option congue nihil imperdiet doming id quod mazim placerat facer possim assum. Lorem ipsum dolor sit amet, consectetuer adipiscing elit, sed diam nonummy nibh euismod tincidunt ut laoreet dolore magna aliquam erat volutpat. Ut wisi enim ad minim veniam, quis nostrud exerci tation ullamcorper suscipit lobortis nisl ut aliquip ex ea commodo consequat. 

Duis autem vel eum iriure dolor in hendrerit in vulputate velit esse molestie consequat, vel illum dolore eu feugiat nulla facilisis.

% section lorem_ipsum_dolor (end)

\section{At Vero eos et accusam} % (fold)
\label{sec:at_vero_eos_et_accusam}

At vero eos et accusam et justo duo dolores et ea rebum. Stet clita kasd gubergren, no sea takimata sanctus est Lorem ipsum dolor sit amet. Lorem ipsum dolor sit amet, consetetur sadipscing elitr, sed diam nonumy eirmod tempor invidunt ut labore et dolore magna aliquyam erat, sed diam voluptua. At vero eos et accusam et justo duo dolores et ea rebum. Stet clita kasd gubergren, no sea takimata sanctus est Lorem ipsum dolor sit amet. Lorem ipsum dolor sit amet, consetetur sadipscing elitr, At accusam aliquyam diam diam dolore dolores duo eirmod eos erat, et nonumy sed tempor et et invidunt justo labore Stet clita ea et gubergren, kasd magna no rebum. sanctus sea sed takimata ut vero voluptua. est Lorem ipsum dolor sit amet. Lorem ipsum dolor sit amet, consetetur sadipscing elitr, sed diam nonumy eirmod tempor invidunt ut labore et dolore magna aliquyam erat.

\begin{lstlisting}[language=C]
/* This does not make algorithmic sense, but it shows off
*  significant programming characters.  */

#include<stdio.h>

void myFunction( int input, float* output ) {
	switch ( array[i] ) {
		case 1: // This is silly code
		if ( a >= 0 || b <= 3 && c != x )
	}
	*output += 0.005 + 20050;
	char = 'g';
	b = 2^n + ~right_size - leftSize * MAX_SIZE;
	c = (--aaa + &daa) / (bbb++ - ccc % 2 );
	strcpy(a,"hello $@?");
	count = ~mask | 0x00FF00AA;
}
\end{lstlisting}

Consetetur sadipscing elitr, sed diam nonumy eirmod tempor invidunt ut labore et dolore magna aliquyam erat, sed diam voluptua. At vero eos et accusam et justo duo dolores et ea rebum. Stet clita kasd gubergren, no sea takimata sanctus est Lorem ipsum dolor sit amet. Lorem ipsum dolor sit amet, consetetur sadipscing elitr, sed diam nonumy eirmod tempor invidunt ut labore et dolore magna aliquyam erat, sed diam voluptua. At vero eos et accusam et justo duo dolores et ea rebum. Stet clita kasd gubergren, no sea takimata sanctus est Lorem ipsum dolor sit amet. Lorem ipsum dolor sit amet, consetetur sadipscing elitr, sed diam nonumy eirmod tempor invidunt ut labore et dolore magna aliquyam erat, sed diam voluptua. At vero eos et accusam et justo duo dolores et ea rebum. Stet clita kasd gubergren, no sea takimata sanctus.

\begin{equation}
	\underbrace{\frac{\sin^{2}\vartheta}{\Theta_{lm}(\vartheta)}\left(\frac{\partial^{2}}{\partial\vartheta^{2}}+\frac{\cos\vartheta}{\sin\vartheta}\frac{\partial}{\partial\vartheta}\right)\Theta_{lm}(\vartheta)+\sin^{2}(\vartheta)(l(l+1))}_{m^{2}}=\underbrace{-\frac{1}{\Phi_{m}(\varphi)}\frac{\partial^{2}}{\partial\varphi^{2}}\Phi_{m}(\varphi)}_{m^{2}}
\end{equation}

\begin{equation}
	P_l (x)\equiv\frac {1}{2^l}\sum_{k=0}^{\lfloor l/2\rfloor} (-1)^k \frac{(2l-2k)!}{k!(l-k)!(l-2k)!} x^{l-2k}
\end{equation}

Lorem ipsum dolor sit amet, consetetur sadipscing elitr, sed diam nonumy eirmod tempor invidunt ut labore et dolore magna aliquyam erat, sed diam voluptua. At vero eos et accusam et justo duo dolores et ea rebum. Stet clita kasd gubergren, no sea takimata sanctus est Lorem ipsum dolor sit amet. Lorem ipsum dolor sit amet, consetetur sadipscing elitr, sed diam nonumy eirmod tempor invidunt ut labore et dolore magna aliquyam erat, sed diam voluptua. At vero eos et accusam et justo duo dolores et ea rebum. Stet clita kasd gubergren, no sea takimata sanctus est Lorem ipsum dolor sit amet. Lorem ipsum dolor sit amet, consetetur sadipscing elitr, sed diam nonumy eirmod tempor invidunt ut labore et dolore magna aliquyam erat, sed diam voluptua. At vero eos et accusam et justo duo dolores et ea rebum. Stet clita kasd gubergren, no sea takimata sanctus est Lorem ipsum dolor sit amet. 

Duis autem vel eum iriure dolor in hendrerit in vulputate velit esse molestie consequat, vel illum dolore eu feugiat nulla facilisis at vero eros et accumsan et iusto odio dignissim qui blandit praesent luptatum zzril delenit augue duis dolore te feugait nulla facilisi. Lorem ipsum dolor sit amet, consectetuer adipiscing elit, sed diam nonummy nibh euismod tincidunt ut laoreet dolore magna aliquam erat volutpat.

% section at_vero_eos_et_accusam (end)

% chapter theoretische_grundlagen (end)
%--------------------------------------------------------------------------------

\chapter{Material und Methoden} % (fold)
\label{cha:material_und_methoden}

% chapter material_und_methoden (end)
%--------------------------------------------------------------------------------

\chapter{Ergebnisse und Diskussion} % (fold)
\label{cha:ergebnisse_und_diskussion}

% chapter ergebnisse_und_diskussion (end)
%--------------------------------------------------------------------------------

\chapter{Zusammenfassung und Ausblick} % (fold)
\label{cha:zusammenfassung_und_ausblick}

% chapter zusammen_und_ausblick (end)

%--------------------------------------------------------------------------------
%	LITERATURVERZEICHNIS - ANHANG
%--------------------------------------------------------------------------------
\bibliographystyle{../chicago-de}
\bibliography{../literatur}

\appendix

\listoffigures

\listoftables
%--------------------------------------------------------------------------------
\end{document}