
\documentclass[%
	BCOR12mm, % Bindekorrektur 12mm (geeignet für Lochung oder einfache Klebebindung)
	cleardoublepage=empty, % Vakatseiten ohne Kopf- und Fußzeile
	tablecaptionabove, % Tabellenüberschriften, aber Bildunterschriften
	toc=bib, % Literaturverzeichnis im Inhaltsverzeichnis aufführen
	toc=listofnumbered, % Abbildungs- und Tabellenverzeichnis im Inhaltverzeichnis aufführen
	listof=leveldown, % Abbildungs- und Tabellenverzeichnis Überschriften als Section statt Chapter
	numbers=noendperiod % Kein abschließender Punkt bei Kapitel-Zählung
	]{scrbook}

\usepackage[utf8]{inputenc} % UTF-8 Zeichen sind möglich (z.B. ö, ü, ä, ...)
\usepackage[ngerman]{babel} % für deutsche Texte

\usepackage{amssymb,amsmath} % Matheumgebung, siehe "mathe_in_latex.pdf"
\usepackage{siunitx} % siehe "siunitx_Einheiten.pdf"
\usepackage{mhchem} % chemische Summenformeln und Gleichungen siehe "Summenformeln und Chemie allgemein mit LaTeX.pdf"
\usepackage{chemfig} % chemische Sturkturformeln darstellen siehe "Strukturformeln mit chemfig.pdf"
\usepackage{pgfplots} % Diagramme pltten siehe "pgfplots_Diagramme.pdf"
\usepackage{tikz} % zeichnen, Deckblattgestaltung, etc. siehe "tikzpgfmanual.pdf"
\usetikzlibrary{backgrounds}
\usepackage{hyperref}
\usepackage{color}

\usepackage{natbib} %Literaturverzeichnis

%% ----------------------------------
%% |  Style of appendix numbering   |
%% ----------------------------------
\renewcommand\appendix{\par
	\addchap{\appendixname}
	\setcounter{section}{0}% 
	\setcounter{subsection}{0}% 
	\setcounter{figure}{0}%
	\renewcommand\thesection{\Alph{section}}% 
	\renewcommand\thefigure{\Alph{section}.\arabic{figure}} 
	\renewcommand\thetable{\Alph{section}.\arabic{table}}}
%% --- End of appenix numbering style ---

%--------------------------------------------------------------------------------
\begin{document}
%--------------------------------------------------------------------------------
%	DECKBLATT
%--------------------------------------------------------------------------------
%!TEX root = /Users/Cornelia/Documents/LaTex/Vorlagen/LaTeX-Vorlage-KIT-Abschlussarbeit/Arbeit/main.tex

%--------------------------------------------------------------------------------
%	DECKBLATT
%--------------------------------------------------------------------------------

\definecolor{grau}{rgb}{0.85,0.85,0.85} %RGB: 217R 217G 217B



\newcommand{\diameter}{5}
\newcommand{\xone}{-20} % -30 wenn keine Bindung/Lochung vorgesehen ist
\newcommand{\xtwo}{164}
\newcommand{\yone}{28}
\newcommand{\ytwo}{-251} % -259 wenn kein Text in die Fußzeile soll

\begin{tikzpicture}[ 
	show background rectangle, 
	background rectangle/.style={fill=grau},
	remember picture, 
	overlay]
	
\fill[color=white](\xone mm, \yone mm)  -- (\xtwo mm, \yone mm) arc (90:0:\diameter mm)   -- (\xtwo mm + \diameter mm , \ytwo mm) 	-- (\xone mm + \diameter mm , \ytwo mm) arc (270:180:\diameter mm)	-- (\xone mm, \yone mm);

\node[xshift=38.5mm,yshift=-17.5mm] at (current page.north west){
	\includegraphics[width=33mm, height=15mm]{../logos/KITLogo_RGB.pdf}
	};
\node[xshift=-26.5mm,yshift=-17.5mm] at (current page.north east){
	\includegraphics[height=15mm]{../logos/LVTLogo.pdf}
	};		 
\node [xshift=5mm,yshift=5mm] at (current page.center)[text width=\textwidth]
	{\titlehead{\Large \center Karlsruher Institut für Technologie (KIT)} % Universität, Institut, ...
	\subject{Abschlussarbeit} % Art der Arbeit (Bachelor-, Master-, Diplomarbeit, Dissertation)
	\title{Titel}
	\subtitle{Untertitel}
	\author{Autor}
	% \publishers{Aufgabensteller \\Betreuer}  % Aufgabensteller / Betreuer
	%\date{} %Datum wird automatisch auf \today gesetzt - für anderes Datum in die {} schreiben
	\maketitle };
	
\node [xshift=90mm,yshift=6.5mm] at (current page.south west)[text width=\textwidth]
	{Karlsruher Institut für Technologie (KIT)};
\node [xshift=42mm,yshift=6.5mm] at (current page.south east)[text width=\textwidth]
	{www.kit.edu};
\end{tikzpicture}
%--------------------------------------------------------------------------------
%--------------------------------------------------------------------------------
%	ERKLÄRUNG - INHALTSVERZEICHNIS - SYMBOLVERZEICHNIS
%--------------------------------------------------------------------------------
\frontmatter
%--------------------------------------------------------------------------------
% \chapter{Erklärung} % (fold)
% \label{cha:erklaerung}

% chapter erklärung (end)
%--------------------------------------------------------------------------------
\tableofcontents
%--------------------------------------------------------------------------------
\chapter{Symbolverzeichnis} % (fold)
\label{cha:symbolverzeichnis}

% chapter symbolverzeichnis (end)

%--------------------------------------------------------------------------------
%	INHALT
%--------------------------------------------------------------------------------
\mainmatter
%--------------------------------------------------------------------------------
% \chapter{Einleitung} % (fold)
% \label{cha:einleitung}

% chapter einleitung (end)
%--------------------------------------------------------------------------------

\chapter{Theoretische Grundlagen} % (fold)
\label{cha:theoretische_grundlagen}

\section{Beispiele} % (fold)
\label{sec:beispiele}

% section beispiele (end)

\subsection*{Ein Diagramm} % (fold)
\label{sub:ein_diagramm}

\begin{tikzpicture}
	\begin{axis}[]
		\addplot table[x index=0, y index=1]{Data/tabelle.txt};
	\end{axis}
\end{tikzpicture}
% subsection ein_diagramm (end)
   
\subsection*{Chemische Formeln} % (fold)
\label{sub:chemische_formeln}

    \ce{H2O} \\
    \ce{H2_{(aq)}}\\
    \ce{$A$ ->[\ce{+H2O}] $B$}\\
    \chemfig{H_3C-{{(CH_2)}_2}-CH_3}\\
    \chemfig{H_3C-C(=[1]O)-[7]O^\ominus}\\
    \chemname{\chemfig{*6(-(<[::-120]H)(*6(-(-[::-20]H_3C)%
(-[::-70]H_3C)-O-(*6(-=(--[:30]-[:-30]-[:30]-%
[:-30]CH_3)-=(-OH)-=))--(<:[::-120]H)-))--=(-CH_3)--)}}{THC}\\
% subsection chemische_formeln (end)


\subsection*{Eine Tabelle} % (fold)
\label{sub:eine_tabelle}
\begin{tabular}{l|rrr}
	 & Spalte 1 & Spalte 2 & Spalte 3 \\
	\hline
	erste Zeile & 1 & 2 & 3 \\
	zweite Zeile & 1 & 2 & 3 \\

\end{tabular}
% subsection eine_tabelle (end)

\subsection*{Zitate} % (fold)
\label{sub:zitate}


Ich zitiere hier \cite{gaukel2004untersuchungen} und \cite{tscheuschner2004grundzuge}. Die Quellen findet man am Ende der Arbeit im Literaturverzeichnis. Erstellt mit BibDesk.
% subsection zitate (end)


% chapter theoretische_grundlagen (end)
%--------------------------------------------------------------------------------

\chapter{Material und Methoden} % (fold)
\label{cha:material_und_methoden}

% chapter material_und_methoden (end)
%--------------------------------------------------------------------------------

\chapter{Ergebnisse und Diskussion} % (fold)
\label{cha:ergebnisse_und_diskussion}

% chapter ergebnisse_und_diskussion (end)
%--------------------------------------------------------------------------------

\chapter{Zusammenfassung und Ausblick} % (fold)
\label{cha:zusammenfassung_und_ausblick}

% chapter zusammen_und_ausblick (end)

%------------------------------------------------------------------------------
%	LITERATURVERZEICHNIS - ANHANG
%------------------------------------------------------------------------------
\bibliographystyle{../chicago-de}
\bibliography{../literatur}

\appendix

\listoffigures

\listoftables
%--------------------------------------------------------------------------------
\end{document}
